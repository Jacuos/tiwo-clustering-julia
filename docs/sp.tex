\documentclass[a4paper,11pt,notitlepage]{article}
\usepackage[utf8]{inputenc}	% latin2 - kodowanie iso-8859-2; cp1250 - kodowanie windows
\usepackage[T1]{fontenc}
\usepackage[polish]{babel}
\usepackage[MeX]{polski}
\selectlanguage{polish}

\usepackage{graphicx}
\usepackage{hyperref}

\hyphenation{FreeBSD}

\author{Mateusz Bocheński\\Paweł Kłapsa\\Jacek Kozieja\\Mateusz Stefaniak}
\title{Testowanie i weryfikacja oprogramowania \\ {\small Projekt 3 - Analiza skupień}}
\date{\today}

\linespread{1.3}

\usepackage{indentfirst}

\begin{document}
\maketitle
\tableofcontents

\section{Wstęp}
Celem projektu było przetestowanie gotowych implementacji algorytmów "Analizy skupień" (Data Clustering). W ramach projektu studenci powinni zapoznać się z językiem programowania \textbf{Julia} a szczególnie z sposobem tworzenia testów.


\section{Opis technologii}
Do realizacji projektu wykorzstany został język \href{http://julialang.org/}{\textbf{Julia}} oraz dedykowana mu paczka \href{http://clusteringjl.readthedocs.org/en/latest/index.html}{\textbf{Clustering}} zawierająca gotowe implementacje funkcji klasteryzacji.


\section{Testy}
Testy

\section{Wnioski}
Język \textbf{Julia} pomimo bycia w początkowej fazie rozwoju (wersja 0.4.2) zawiera wiele gotowych paczek z implementacjami różnorodnych algorytmów. Jego twórcy kładą szczególny nacisk na wydajność, dzięki czemu istnieje spore prawdopodobieństwo że w przyszłości wyprze on takie języki jak \textbf{MATLAB} czy \textbf{Python} wykorzystywane przy wykonywaniu złożonych obliczeń. Język ten ma także bardzo podobną składnię do języka \textbf{MATLAB} co znacznie ułatwia rozpoczęcie z nim pracy.

\end{document}
